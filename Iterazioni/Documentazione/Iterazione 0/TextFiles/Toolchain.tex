\section{Toolchain e tecnologie utilizzate}
Per la realizzazione del sistema verranno utilizzati i seguenti strumenti:
\begin{itemize}
	\item \textbf{Modellazione}:
	\begin{itemize}
		\item diagramma dei casi d'uso, deployment diagram, component diagram, class
		diagram, sequence diagram, diagramma entità-relazione: Visio.
	\end{itemize}
	\item \textbf{Implementazione Applicazione Client}:
	\begin{itemize}
		\item Linguaggio di programmazione: C++.
		\item IDE: Qt Creator 6.0.2.
		\item Interfaccia grafica: QML.
		\item Analisi statica: CppCheck.
		\item Analisi dinamica: Qt Test.
	\end{itemize}
	\item \textbf{Implementazione Web Server}:
	\begin{itemize}
		\item Linguaggio di programmazione: Java.
		\item IDE: Eclipse.
		\item Framework: Spring.
		\item Analisi statica: STAN4J.
		\item Analisi dinamica: JUnit.
	\end{itemize}
	\item \textbf{Implementazione Data Analyzer e Data Fetcher}:
	\begin{itemize}
		\item Linguaggio di programmazione: python.
		\item IDE: DA DECIDERE.
	\end{itemize}
	\item \textbf{Implementazione Database}:
	\begin{itemize}
		\item Tipologia: relazionale.
		\item Database: MySQL.
		\item Provider: Azure.
	\end{itemize}
	\item \textbf{Documentazione, versioning e organizzazione del team}:
	\begin{itemize}
		\item Documentazione: Latex, con scrittura tramite editor TexStudio.
		\item Versioning: GitHub.
		\item Git client: Sourcetree.
		\item Organizzazione del Team: Microsoft Teams.
	\end{itemize}
\end{itemize}