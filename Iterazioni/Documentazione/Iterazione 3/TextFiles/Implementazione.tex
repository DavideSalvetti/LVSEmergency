\section{Introduzione}
Nella terza iterazione sono stati implementati i componenti \textit{Data Analyzer} e \textit{Data collector}. \todo{Spiegare un attimo a cosa servono velocemente}

Grazie all'implementazione di questi due componenti, è stato possibile realizzare anche i seguenti casi d'uso:
\begin{itemize}
	\item UC18 - Visualizzazione informazioni zona [astratto];
	\begin{itemize}
		\item UC18.1 - Visualizzazione dati ambientali;
		\item UC18.1 - Visualizzazione allarmi;
	\end{itemize}
\end{itemize}


\subsection{Data Collector}
\todo{Spiegare cosa fa, come è stato implementato, linguaggio utilizzato.}

\import{./Iterazione 3/TextFiles}{EsempioAPIAPRS.tex}


\subsection{Data Analyzer}
\todo{Spiegare cosa fa, come è stato implementato, linguaggio utilizzato. Spiegare anche l'algoritmo. Ricordarsi che bisogna mettere i test dell'algoritmo}

\subsection{UC18.1 - Visualizzazione dati ambientali}
\textit{Breve descrizione:} il volontario (o caposquadra) accede alla pagina in cui gli vengono mostrati gli ultimi dati disponibili relativi all'area del team di cui fa parte. 
\\
\\
\textit{Attori coinvolti:} volontario, caposquadra, sistema.
\\
\\
\textit{Precondizione:} il volontario (o caposquadra) è registrato nel sistema.
\\
\\
\textit{Postcondizione:} il volontario (o caposquadra) visualizza i dati relativi all'area del team di cui fa parte.
\\
\\
\textit{Procedimento:}
\begin{enumerate}
	\item il volontario (o caposquadra) accede alla pagina "Dati Area".
	\item il sistema mostra gli ultimi dati disponibili relativi a:
	\begin{itemize}
		\item stazioni APRS;
		\item terremoti;
		\item bollettini della protezione civile.
	\end{itemize}
\end{enumerate}

\subsection{UC18.2 - Visualizzazione allarmi}
\textit{Breve descrizione:} il volontario (o caposquadra) accede alla pagina in cui gli vengono mostrati gli allarmi disponibili relativi all'area del team di cui fa parte. 
\\
\\
\textit{Attori coinvolti:} volontario, caposquadra, sistema.
\\
\\
\textit{Precondizione:} il volontario (o caposquadra) è registrato nel sistema.
\\
\\
\textit{Postcondizione:} il volontario (o caposquadra) visualizza gli allarmi relativi all'area del team di cui fa parte.
\\
\\
\textit{Procedimento:}
\begin{enumerate}
	\item il volontario (o caposquadra) accede alla pagina "Dati Area".
	\item il sistema mostra gli allarmi emessi per la specifica area:
	\begin{itemize}
		\item allarme nebbia o brina;
		\item allarme maltempo;
	\end{itemize}
\end{enumerate}
