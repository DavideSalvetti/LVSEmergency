\section{Manuale Utente}
Grazie alle scelte tecnologiche con cui sono stati implementati i diversi componenti, il prodotto software può essere utilizzato su qualunque sistema operativo, sia desktop (Windows, MacOs, Linux), sia mobile (Android o iOs).\todo{Davide chiede: "Si scrive così iOs?"}
Infatti, la particolarità del framework Qt utilizzato per lo sviluppo dell'applicazione client risiede nel fatto che, con un unico codice sorgente, è possibile creare applicazioni in grado di girare su una gran parte dei sistemi operativi in commercio. 
Inoltre, il server Spring è in esecuzione su una istanza di Azure e di conseguenza è sempre accessibile dall'applicazione client.
Il database è anch'esso su un'istanza MySQL di Azure ed è accessibile in qualunque momento.
Invece, il data collector ed il data analyzer sono in esecuzione su un Raspberry Pi 4.
Questo significa che l'utente deve semplicemente scaricare ed installare l'apk su un telefono oppure scaricare e lanciare l'eseguibile su un PC.
\todo{Da rileggere}

\section{Sviluppi futuri}
Non tutti i casi d'uso definiti sono stati implementati. Di seguito è riportata una tabella riassuntiva di quali casi d'uso sono stati implementati e quali no:

\begin{center}
	\begin{tabular}{|c|c|c|}
		\hline
		\textbf{Codice} & \textbf{Caso d'uso} & \textbf{Implementato} \\ \hline
		\multicolumn{3}{|c|}{Alta Priorità} \\ \hline
		\textbf{UC1} & Login & Sì\\ \hline
		\textbf{UC2} & Logout & Sì \\ \hline
		\textbf{UC3} & Visualizzazione informazioni account & Sì \\ \hline
		\textbf{UC4} & Visualizzazione informazioni squadra & Sì \\ \hline
		\textbf{UC15} & Inserimento utente & Sì \\ \hline
		\textbf{UC16} & Cancellazione utente & Sì\\ \hline
		\textbf{UC17} & Gestione squadre (creazione) & Sì \\ \hline
		\textbf{UC18} & Visualizzazione informazioni zona & Sì \\ \hline
		\multicolumn{3}{|c|}{Media Priorità} \\ \hline
		\textbf{UC6} & Segnalazione operatività & Sì\\ \hline
		\textbf{UC7} & Visualizzazione intervento di emergenza & No \\ \hline
		\textbf{UC8} & Visualizzazione intervento programmato & No \\ \hline
		\textbf{UC9} & Inserimento informazioni intervento & No \\ \hline
		\textbf{UC11} & Gestione intervento di emergenza & No \\ \hline
		\textbf{UC12} & Gestione intervento programmato & No \\ \hline
		\textbf{UC13} & Gestione report intervento di emergenza & No \\ \hline
		\textbf{UC14} & Gestione report intervento programmato & No \\ \hline
		\textbf{UC20} & Gestione informazioni relative alla zona & No \\ \hline
		\multicolumn{3}{|c|}{Bassa Priorità} \\ \hline
		\textbf{UC5} & Gestione reperibilità & No\\ \hline
		\textbf{UC10} & Visualizzazione posizione real-time & No\\ \hline
		\textbf{UC19} & Notifiche allarmi zona & No\\ \hline
	\end{tabular}
\end{center}
 