\section{Introduzione}
Nella seconda iterazione sono stati implementati i seguenti casi d'uso:
\begin{itemize}
	\item UC1 - Login;
	\item UC2 - Logout;
	\item UC3 - Visualizzazione informazioni account;
	\item UC4 - Visualizzazione informazioni squadra;
	\item UC6 - Segnalazione operatività;
	\item UC15 - Inserimento utente;
	\item UC16 - Cancellazione utente;
	\item UC17 - Gestione squadre [astratto]:
	\begin{itemize}
		\item UC17.1 - Creazione squadra;
	\end{itemize}
\end{itemize}

\subsection{UC1 - Login}
Per l'autenticazione è stato utilizzato il metodo \textit{Basic Access Authentication}: si tratta di una tecnica che non necessita dell'utilizzo di cookie o di mentenere una sessione tra client e server, ma utilizza gli \textit{header} HTTP per fornire le informazioni di accesso. In particolare, \textit{username} e \textit{password} vengono codificate con \textit{base64} e vengono trasmesse nell'\textit{header} ogni volta che viene chiamata una API. Il sistema poi verifica che lo \textit{username} e la \textit{password} presenti nell'\textit{header} HTTP appartengano effettivamente ad un utente presente nel database prima di elaborare la richiesta. 
\\
\\
\textit{Breve descrizione:} l'utente compila il form per eseguire il login: in caso di credenziali corrette il sistema consente l'accesso ai servizi, altrimenti notifica l'utente della non correttezza delle credenziali. 
\\
\\
\textit{Attori coinvolti:} volontario, caposquadra, coordinatore, sistema.
\\
\\
\textit{Precondizione:} l'utente è registrato nel sistema.
\\
\\
\textit{Postcondizione:} l'utente accede alla \textit{dashboard} (in caso le credenziali siano corrette) oppure viene notificato dell'incorrettezza delle credenziali.
\\
\\
\textit{Procedimento:}
\begin{enumerate}
	\item il sistema richiede all'utente le informazioni di accesso (username e password).
	\item l'utente inserisce le informazioni di accesso.
	\item il sistema controlla le informazioni fornite.
	\item le informazioni sono corrette. [E1: le informazioni sono sbagliate].
	\item l'utente accede alla \textit{dashboard}.
\end{enumerate}


\textit{Eccezioni:}
\begin{itemize}
	\item E1:
	\begin{enumerate}
		\item le informazioni sono sbagliate.
		\item il sistema comunica all'utente che le informazioni inserite non sono corrette.
		\item ritorno al passo 1 del "Procedimento".
	\end{enumerate}
\end{itemize}



\subsection{UC2 - Logout}
Grazie alla \textit{Base Access Authentication}, il logout dell'utente consiste nel rimandarlo alla pagina di login per richiedere nuovamente \textit{username} e \textit{password}. Infatti, non c'è nessuna sessione tra client e server, quindi questo caso d'uso viene gestito lato client chiedendo all'utente di autenticarsi di nuovo.
\\
\\ 
\textit{Breve descrizione:} il sistema esegue il logout dell'utente.
\\
\\
\textit{Attori coinvolti:} volontario, caposquadra, coordinatore, sistema.
\\
\\
\textit{Precondizione:} l'utente è loggato nel sistema.
\\
\\
\textit{Postcondizione:} l'utente viene reindirizzato alla pagina di login e non è più loggato nel sistema.
\\
\\
\textit{Procedimento:}
\begin{enumerate}
	\item l'utente richiede al sistema di eseguire il logout.
	\item il sistema effettua il logout dell'utente.
\end{enumerate}

\subsection{UC3 - Visualizzazione informazioni account}
\textit{Breve descrizione:} l'utente visualizza le informazioni del suo account.
\\
\\
\textit{Attori coinvolti:} volontario, caposquadra, coordinatore, sistema. 
\\
\\
\textit{Precondizione:} l'utente è loggato nel sistema. L'utente è nella pagina "Informazioni".
\\
\\
\textit{Postcondizione:} il sistema mostra le informazioni dell'account dell'utente.
\\
\\
\textit{Procedimento:}
\begin{enumerate}
	\item il sistema mostra le informazioni dell'account dell'utente.
\end{enumerate}


\subsection{UC4 - Visualizzazione informazioni squadra}
\textit{Breve descrizione:} l'utente visualizza le informazioni della squadra a cui è stato assegnato.
\\
\\
\textit{Attori coinvolti:} volontario, caposquadra, sistema.
\\
\\
\textit{Precondizione:} l'utente è loggato nel sistema. L'utente è nella pagina "Informazioni".
\\
\\
\textit{Postcondizione:} il sistema mostra le informazioni della squadra a cui l'utente è assegnato.
\\
\\
\textit{Procedimento:}
\begin{enumerate}
	\item il sistema mostra le informazioni della squadra a cui l'utente è assegnato tra cui:
	\begin{itemize}
		\item Nome squadra;
		\item Nome area;
		\item Numero dei membri della squadra;
		\item Informazioni sugli utenti appartenenti alla squadra. Per ogni utente viene mostrato:
		\begin{itemize}
			\item Nome;
			\item Cognome;
			\item Ruolo;
			\item Cellulare;
			\item Email.
		\end{itemize}
	\end{itemize}
\end{enumerate}

\subsection{UC6 - Segnalazione operatività}
\textit{Breve descrizione:} l'utente segnala lo stato di operatività ("Attivo" o "Inattivo"). 
\\
\\
\textit{Attori coinvolti:} volontario, caposquadra, sistema.
\\
\\
\textit{Precondizione:} l'utente è loggato nel sistema. L'utente è nella "Dashboard".
\\
\\
\textit{Postcondizione:} il sistema salva lo stato di operatività dell'utente.
\\
\\
\textit{Procedimento:}
\begin{enumerate}
	\item l'utente clicca sul pulsante per cambiare il suo stato di operatività.
	\item il sistema salva lo stato di operatività e lo mostra all'utente.
\end{enumerate}



\subsection{UC15 - Inserimento utente}
\textit{Breve descrizione:} il coordinatore compila il form per inserire un nuovo volontario di cui ha ricevuto una richiesta di iscrizione. 
\\
\\
\textit{Attori coinvolti:} coordinatore, sistema.
\\
\\
\textit{Precondizione:} il coordinatore è loggato nel sistema. Il coordinatore è nella pagina "Inserisci utente".
\\
\\
\textit{Postcondizione:} il volontario viene inserito nel database e può accedere ai servizi.
\\
\\
\textit{Procedimento:}
\begin{enumerate}
	\item il coordinatore fornisce le seguenti informazioni nel form per la registrazione di un nuovo volontario:
	\begin{itemize}
		\item Username;
		\item Nome;
		\item Cognome;
		\item Password;
		\item Sesso;
		\item Codice fiscale;
		\item Indirizzo;
		\item Cellulare;
		\item Email;
		\item Ruolo: nel caso il ruolo selezionato sia "Caposquadra" verranno mostrate solo le squadre a cui non è ancora stato assegnato un caposquadra;
		\item Squadra.
	\end{itemize}
	\item il coordinatore clicca il pulsante "Inserisci utente".
	\item il sistema verifica che tutti i campi siano stati compilati [E1; ci sono dei campi vuoti].
	\item il sistema verifica che username e codice fiscale non siano già associati ad altri utenti.
	\item il sistema aggiunge l'utente con i rispettivi dati nel database [E2: i dati sono già stati associati ad un altro utente presente nel database].
	\item se l'utente inserito ha il ruolo di "Caposquadra", il sistema imposta l'utente come caposquadra della squadra selezionata.
\end{enumerate}


\textit{Eccezioni:}
\begin{itemize}
	\item E1:
	\begin{enumerate}
		\item i dati sono già associati ad un altro utente presente nel database.
		\item il sistema comunica al coordinatore che i dati sono già associati ad un altro utente.
		\item ritorno al passo 1 del "Procedimento".
	\end{enumerate}
	\item E2:
	\begin{enumerate}
		\item i dati sono già associati ad un altro utente presente nel database.
		\item il sistema comunica al coordinatore che i dati sono già associati ad un altro utente.
		\item ritorno al passo 1 del "Procedimento".
	\end{enumerate}
\end{itemize}

\subsection{UC16 - Cancellazione utente}
\textit{Breve descrizione:} il coordinatore cancella un utente dal sistema. 
\\
\\
\textit{Attori coinvolti:} coordinatore, sistema.
\\
\\
\textit{Precondizione:} il coordinatore è loggato nel sistema.
L'utente che deve essere cancellato è inserito nel sistema.
Il coordinatore è nella pagina "Cancella utente".
\\
\\
\textit{Postcondizione:} l'utente viene cancellato dal sistema.
\\
\\
\textit{Procedimento:}
\begin{enumerate}
	\item il sistema mostra tutti gli utenti registrati tranne i coordinatori che non possono essere cancellati per motivi di sicurezza.
	\item il coordinatore preme il pulsante per cancellare un utente.
	\item l'utente viene cancellato.
\end{enumerate}

\subsection{UC17.1 - Creazione squadra}
\textit{Breve descrizione:} il coordinatore crea una nuova squadra. 
\\
\\
\textit{Attori coinvolti:} coordinatore, sistema.
\\
\\
\textit{Precondizione:} il coordinatore è loggato nel sistema.
\\
\\
\textit{Postcondizione:} la nuova squadra è inserita nel database.
\\
\\
\textit{Procedimento:}
\begin{enumerate}
	\item il coordinatore inserisce le informazioni richieste per la creazione di una squadra:
	\begin{itemize}
		\item il nome della squadra.
		\item l'area di competenza della squadra.
	\end{itemize}
	\item il coordinatore preme il pulsante "Crea squadra".
	\item il sistema verifica che siano stati inseriti tutti i campi [E1: ci sono dei campi vuoti].
	\item il sistema verifica che il nome della squadra non sia già stato assegnato ad altre squadre [E2: il nome della squadra è già stato assegnato ad un'altra squadra].
	\item il sistema inserisce la nuova squadra nel database. 
\end{enumerate}

\textit{Eccezioni:}
\begin{itemize}
	\item E1:
	\begin{enumerate}
		\item ci sono dei campi vuoti.
		\item il sistema comunica al coordinatore che ci sono dei campi vuoti.
		\item ritorno al passo 1 del "Procedimento".
	\end{enumerate}
	\item E2:
	\begin{enumerate}
		\item il nome della squadra è già stato assegnato.
		\item il sistema comunica al coordinatore che il nome della squadra è già in uso.
		\item ritorno al passo 1 del "Procedimento".
	\end{enumerate}
\end{itemize}


