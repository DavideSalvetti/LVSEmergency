\section{Implementazione}
Nella prima iterazione \todo{o forse seconda, bisogna decidere dove inserirla alla luce del fatto che nella prima iterazione abbiamo già inserito tutti i diagrammi e qui dovremmo specificare i casi d'uso e tutti i diagrammi credo.} sono stati implementati i seguenti casi d'uso:
\begin{itemize}
	\item UC1 - Login;
	\item UC2 - Logout;
	\item UC15 - Inserimento volontario;
	\item UC16 - Cancellazione volontario;
	\item UC17 - Gestione squadre [Astratto]:
	\begin{itemize}
		\item UC17.1 - Creazione squadra;
		\item UC17.2 - Modifica squadra;
		\item UC17.3 - Elimina squadra;
		\item UC17.4 - Visualizza squadra.
	\end{itemize}
\end{itemize}

\subsection{UC1 - Login}
\textit{Breve descrizione:} l'utente compila il form per eseguire il login: in caso di credenziali corrette il sistema consente l'accesso ai servizi, altrimenti notifica l'utente della non correttezza delle credenziali. 
\\
\\
\textit{Attori coinvolti:} volontario, caposquadra, coordinatore, sistema.
\\
\\
\textit{Trigger:}
\\
\\
\textit{Precondizione:} l'utente è registrato nel sistema.
\\
\\
\textit{Postcondizione:} l'utente accede alla \textit{login page} (in caso le credenziali siano corrette) oppure viene notificato dell'incorrettezza delle credenziali.
\\
\\
\textit{Procedimento:}
\begin{enumerate}
	\item il sistema richiede all'utente le informazioni di accesso (email e password).
	\item l'utente inserisce le informazioni di accesso.
	\item il sistema controlla le informazioni fornite.
	\item le informazioni sono corrette. [E1: le informazioni sono sbagliate].
	\item l'utente viene loggato nel sistema.
\end{enumerate}


\textit{Eccezioni:}
\begin{itemize}
	\item E1:
	\begin{enumerate}
		\item le informazioni sono sbagliate.
		\item il sistema comunica all'utente che le informazioni inserite non sono corrette.
		\item ritorno al passo 1 del "Procedimento".
	\end{enumerate}
\end{itemize}



\subsection{UC2 - Logout}
\textit{Breve descrizione:} il sistema esegue il logout dell'utente.
\\
\\
\textit{Attori coinvolti:} volontario, caposquadra, coordinatore, sistema.
\\
\\
\textit{Trigger:}
\\
\\
\textit{Precondizione:} l'utente è loggato nel sistema.
\\
\\
\textit{Postcondizione:} l'utente viene rimandato alla \textit{login page} e non è più loggato nel sistema.
\\
\\
\textit{Procedimento:}
\begin{enumerate}
	\item l'utente richiede al sistema di eseguire il logout.
	\item il sistema effettua il logout dell'utente.
\end{enumerate}


\subsection{UC15 - Inserimento volontario}
\textit{Breve descrizione:} il coordinatore compila il form per inserire un nuovo volontario di cui ha ricevuto una richiesta di iscrizione. 
\\
\\
\textit{Attori coinvolti:} coordinatore, sistema.
\\
\\
\textit{Trigger:}
\\
\\
\textit{Precondizione:} il coordinatore è loggato nel sistema.
\\
\\
\textit{Postcondizione:} il volontario viene inserito nel database e può accedere ai servizi.
\\
\\
\textit{Procedimento:}
\begin{enumerate}
	\item il coordinatore preme il pulsante per aggiungere un nuovo volontario.
	\item il coordinatore fornisce le seguenti informazioni nel form per la registrazione di un nuovo volontario:
	\begin{itemize}
		\item Nome;
		\item Cognome;
		\item Codice Fiscale;
		\item Numero Carta d'identità;
		\item Email;
		\item Numero di telefono;
		\item Password;\todo{è una stupidaggine che inserisca il coordinatore la password, in questo momento è solo ed esclusivamente per semplificare. è ovvio che poi questo sistema non va assolutamente bene. Aggiungere altri campi se necessari, questi sono i primi che mi son venuti in mente ma non sono gli unici}
	\end{itemize}
	\item il sistema verifica che email e codice fiscale non siano già associati ad altri utenti.
	\item il sistema aggiunge il volontario con i rispettivi dati nel database [E1: i dati sono già stati associati ad un altro volontario presente nel database].
\end{enumerate}


\textit{Eccezioni:}
\begin{itemize}
	\item E1:
	\begin{enumerate}
		\item i dati sono già associati ad un altro volontario presente nel database.
		\item il sistema comunica al coordinatore che i dati sono già associati ad un altro volontario.
		\item ritorno al passo 2 del "Procedimento".
	\end{enumerate}
\end{itemize}

\subsection{UC16 - Cancellazione volontario}
\textit{Breve descrizione:} il coordinatore cancella un volontario dal sistema. 
\\
\\
\textit{Attori coinvolti:} coordinatore, sistema.
\\
\\
\textit{Trigger:}
\\
\\
\textit{Precondizione:} il coordinatore è loggato nel sistema.
Il volontario che deve essere cancellato è inserito nel sistema.
\\
\\
\textit{Postcondizione:} il volontario viene cancellato dal sistema.
\\
\\
\textit{Procedimento:}
\begin{enumerate}
	\item il coordinatore preme il pulsante per cancellate un volontario.
	\item il volontario viene cancellato.
\end{enumerate}
\todo{Non ci sono eccezioni perchè il fatto che il volontario sia presente nel sistema è una precondizione. Per come lo faremo noi, il coordinatore scorre tra una lista di volontari e preme il pulsante elimina per eliminarli. Non è possibile quindi eliminare un volontario che non è presente nel sistema.}

\subsection{UC17.1 - Creazione squadra}
\textit{Breve descrizione:} il coordinatore crea una nuova squadra. 
\\
\\
\textit{Attori coinvolti:} coordinatore, sistema.
\\
\\
\textit{Trigger:}
\\
\\
\textit{Precondizione:} il coordinatore è loggato nel sistema.
C'è almeno un caposquadra libero a cui può essere assegnata la squadra.
\\
\\
\textit{Postcondizione:} la nuova squadra è inserita nel database.
\\
\\
\textit{Procedimento:}
\begin{enumerate}
	\item il coordinatore preme il pulsante per creare una nuova squadra.
	\item il coordinatore inserisce le informazioni richieste per la creazione di una squadra, come:
	\begin{itemize}
		\item il nome della squadra.
		\item il caposquadra (da scegliere tra quelli che non hanno già una squadra).
		\item l'area di competenza della squadra.
	\end{itemize}
	\item il sistema verifica che il nome della squadra non sia già stato assegnato ad altre squadre [E1: il nome della squadra è già stato assegnato ad un'altra squadra].
	\item il sistema inserisce la nuova squadra nel database. 
\end{enumerate}

\textit{Eccezioni:}
\begin{itemize}
	\item E1:
	\begin{enumerate}
		\item il nome della squadra è già stato assegnato.
		\item il sistema comunica al coordinatore che il nome della squadra è già in uso.
		\item ritorno al passo 2 del "Procedimento".
	\end{enumerate}
\end{itemize}



