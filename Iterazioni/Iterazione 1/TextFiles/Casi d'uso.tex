\subsection{Specificazione casi d'uso}

\subsection{UC1 - Login}
\textit{Breve descrizione:} l'utente compila il form per eseguire il login: in caso di credenziali corrette il sistema consente l'accesso ai servizi, altrimenti notifica l'utente della non correttezza delle credenziali. 
\\
\\
\textit{Attori coinvolti:} volontario, caposquadra, coordinatore, sistema.
\\
\\
\textit{Precondizione:} l'utente è registrato nel sistema.
\\
\\
\textit{Postcondizione:} se le credenziali sono corrette, l'utente accede alla \textit{dashboard}, altrimenti viene notificato dell'incorrettezza delle credenziali.
\\
\\
\textit{Procedimento:}
\begin{enumerate}
	\item il sistema richiede all'utente le informazioni di accesso (username e password).
	\item l'utente inserisce le informazioni di accesso.
	\item il sistema controlla le informazioni fornite.
	\item le informazioni sono corrette. [E1: le informazioni sono sbagliate].
	\item l'utente viene loggato nel sistema.
\end{enumerate}


\textit{Eccezioni:}
\begin{itemize}
	\item E1:
	\begin{enumerate}
		\item le informazioni sono sbagliate.
		\item il sistema comunica all'utente che le informazioni inserite non sono corrette.
		\item ritorno al passo 1 del "Procedimento".
	\end{enumerate}
\end{itemize}

\subsection{UC2 - Logout}
\textit{Breve descrizione:} il sistema esegue il logout dell'utente.
\\
\\
\textit{Attori coinvolti:} volontario, caposquadra, coordinatore, sistema.
\\
\\
\textit{Precondizione:} l'utente è loggato nel sistema.
\\
\\
\textit{Postcondizione:} l'utente viene rimandato alla \textit{login page} e non è più loggato nel sistema.
\\
\\
\textit{Procedimento:}
\begin{enumerate}
	\item l'utente richiede al sistema di eseguire il logout.
	\item il sistema effettua il logout dell'utente.
\end{enumerate}

\subsection{UC3 - Visualizzazione informazioni account}
\textit{Breve descrizione:} l'utente visualizza le informazioni del suo account.
\\
\\
\textit{Attori coinvolti:} volontario, caposquadra, coordinatore, sistema. 
\\
\\
\textit{Precondizione:} l'utente è loggato nel sistema.
\\
\\
\textit{Postcondizione:} il sistema mostra le informazioni dell'account dell'utente.
\\
\\
\textit{Procedimento:}
\begin{enumerate}
	\item il sistema mostra le informazioni dell'account dell'utente.
\end{enumerate}


\subsection{UC4 - Visualizzazione informazioni squadra}
\textit{Breve descrizione:} l'utente visualizza le informazioni della squadra a cui è stato assegnato.
\\
\\
\textit{Attori coinvolti:} volontario, caposquadra, coordinatore, sistema. L'utente è nella pagina "Informazioni".
\\
\\
\textit{Precondizione:} l'utente è loggato nel sistema. L'utente è nella pagina "Informazioni".
\\
\\
\textit{Postcondizione:} il sistema mostra le informazioni della squadra a cui l'utente è assegnato.
\\
\\
\textit{Procedimento:}
\begin{enumerate}
	\item il sistema mostra le informazioni della squadra a cui l'utente è assegnato.
\end{enumerate}

\subsection{UC6 - Segnalazione operatività}
\textit{Breve descrizione:} l'utente segnala lo stato di operatività. 
\\
\\
\textit{Attori coinvolti:} volontario, caposquadra, sistema.
\\
\\
\textit{Precondizione:} l'utente è loggato nel sistema.
\\
\\
\textit{Postcondizione:} il sistema salva lo stato di operatività dell'utente.
\\
\\
\textit{Procedimento:}
\begin{enumerate}
	\item l'utente seleziona lo stato di operatività (operativo, non operativo).
	\item il sistema salva lo stato di operatività inviato dall'utente.
\end{enumerate}



\subsection{UC15 - Inserimento utente}
\textit{Breve descrizione:} il coordinatore compila il form per inserire un nuovo volontario di cui ha ricevuto una richiesta di iscrizione. 
\\
\\
\textit{Attori coinvolti:} coordinatore, sistema.
\\
\\
\textit{Precondizione:} il coordinatore è loggato nel sistema.
\\
\\
\textit{Postcondizione:} il volontario viene inserito nel database e può accedere ai servizi.
\\
\\
\textit{Procedimento:}
\begin{enumerate}
	\item il coordinatore accede alla pagina per l'inserimento di un nuovo utente.
	\item il coordinatore fornisce le informazioni richieste per l'aggiunta dell'utente.
	\item il sistema verifica che username e codice fiscale non siano già associati ad altri utenti.
	\item il sistema aggiunge l'utente con i rispettivi dati nel database [E1: i dati sono già stati associati ad un altro utente presente nel database].
\end{enumerate}


\textit{Eccezioni:}
\begin{itemize}
	\item E1:
	\begin{enumerate}
		\item i dati sono già associati ad un altro utente presente nel database.
		\item il sistema comunica al coordinatore che i dati sono già associati ad un altro utente.
		\item ritorno al passo 2 del "Procedimento".
	\end{enumerate}
\end{itemize}

\subsection{UC16 - Cancellazione utente}
\textit{Breve descrizione:} il coordinatore cancella un utente dal sistema. 
\\
\\
\textit{Attori coinvolti:} coordinatore, sistema.
\\
\\
\textit{Precondizione:} il coordinatore è loggato nel sistema.
L'utente che deve essere cancellato è inserito nel sistema.
\\
\\
\textit{Postcondizione:} l'utente viene cancellato dal sistema.
\\
\\
\textit{Procedimento:}
\begin{enumerate}
	\item il coordinatore preme il pulsante per cancellare un utente.
	\item l'utente viene cancellato.
\end{enumerate}

\subsection{UC17 - Gestione squadre [astratto]}
\textit{Breve descrizione:} il coordinatore deve essere in grado di aggiungere, modificare, eliminare e visualizzare le squadre. La gestione delle squadre è suddivisa in 4 casi d'uso:
\begin{itemize}
	\item UC17.1 - Creazione squadra;
	\item UC17.2 - Modifica squadra;
	\item UC17.3 - Elimina squadra;
	\item UC17.4 - Visualizza squadra.
\end{itemize}

\textit{Attori coinvolti:} coordinatore, sistema.
\\
\\
\textit{Precondizione:} il coordinatore è loggato nel sistema.
\\
\\
\textit{Postcondizione:} il coordinatore può gestire le squadre scegliendo dal menù di creare, modificare, eliminare o visualizzare le squadre.
\\
\\
\textit{Procedimento:}
\begin{enumerate}
	\item il coordinatore è loggato nel sistema con successo.
	\item il coordinatore clicca il menù.
	\item il sistema mostra le azioni possibili tra cui: "Creazione squadra", "Modifica squadra", "Elimina squadra" e "Visualizza squadre".
\end{enumerate}