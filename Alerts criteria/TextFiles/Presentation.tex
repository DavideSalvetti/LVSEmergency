\subsection{Arrivo di una perturbazione meteorologica} 
Esistono due differenti approcci per prevedere l'andamento delle condizioni meteorologiche a partire dalle rilevazioni della pressione atmosferica $p_{atm}$:
\begin{itemize}
	\item \textit{generale} (analisi della $p_{atm}$ in termini assoluti): se la pressione è inferiore a \SI{1000}{\hecto\pascal}, allora è probabile che il tempo volga brutto; tale probabilità aumenta in presenza di venti meridionali e di un'umidità superiore al \SI{50}{\percent}. Se, invece, la pressione è superiore a \SI{1025}{\hecto\pascal}, allora è probabile che il tempo tenda al bello; tale probabilità aumenta in presenza di venti settentrionali e di un'umidità inferiore al \SI{60}{\percent};
	\item \textit{specifico} (analisi della $p_{atm}$ in termini di variazioni temporali): un calo di 1-\SI{2}{\hecto\pascal} in 3 ore precorre un peggioramento che si manifesta entro le prossime 24-48 ore, una diminuzione superiore a 2-\SI{3}{\hecto\pascal} in 3 ore entro le prossime 12-24 ore, mentre un calo di 5-\SI{6}{\hecto\pascal} in 3 ore sta a indicare un peggioramento imminente per di più da associare a una perturbazione violenta. 
\end{itemize}
La pressione atmosferica, inoltre, è soggetta a un andamento periodico nel corso del giorno:
\begin{itemize}
	\item primo minimo alle ore 4;
	\item primo massimo alle ore 10;
	\item secondo minimo alle ore 16;
	\item secondo massimo alle ore 22.
\end{itemize}
Pertanto, è importante innanzitutto destagionalizzare la serie storica affinché possano essere studiate le variazioni relative esclusivamente a un cambiamento delle condizioni meteorologiche.

\paragraph{Modus operandi} Utilizzando i criteri sopra descritti, è possibile prevedere lo sviluppo del tempo atmosferico con largo anticipo rilevando le variazioni di pressione in una finestra temporale di 3 ore, quindi non si ritene opportuno definire una modello stocastico (es. ARMA) per la $p_{atm}$ a scopo predittivo.

\paragraph{Proposta di algoritmo} Ai temporali sono associati i cumulonembi, ossia nubi a sviluppo verticale all'interno dei quali sono presenti forti correnti ascensionali. L'idea è quella di combinare l'andamento locale della pressione atmosferica, primo campanello d'allarme associato a un cambiamento meteorologico, con l'individuazione di fronti temporaleschi nelle vicinanze della località d'interesse che potrebbero entro un'ora (caso peggiore) sopraggiungere. Il sistema di allerta potrebbe essere realizzato mediante l'implementazione delle seguenti fasi:
\begin{enumerate}
	\item \textit{previsione}: analisi dell'andamento locale della pressione atmosferica. Se la $p_{atm}$ è calata di 5-\SI{6}{\hecto\pascal} nelle 3 ore passate, allora è probabile che un temporale si stia avvicinando, pertanto si passa alla fase successiva;
	\item \textit{individuazione}: analisi di report API contenenti informazioni sulle tipologie di nubi osservate (l'obiettivo sono i cumulonembi), sulla velocità e direzione del vento a \SI{500}{\hecto\pascal} relativi a stazioni meteorologiche dislocate entro un raggio di \SI{80}{\kilo \meter} dalla località d'interesse per capire se effettivamente un fronte temporalesco si sta o meno avvicinando (basandosi sulla direzione del vento) e, in caso di esito positivo, stimare l'orario di arrivo (basandosi sulla velocità del vento).
\end{enumerate}
Studiare i report di tutte le stazioni dislocate entro un raggio di \SI{80}{\kilo\meter} è impossibile, pertanto si potrebbe utilizzare la velocità del vento locale per definire la radiale lungo la quale cercare il fronte temporalesco.\newline

\subsection{Formazione di nebbia e brina}
Per prevedere le formazioni di nebbia e brina è necessario calcolare il \textit{punto di rugiada} $T_d$, ossia la temperatura alla quale l'acqua contenuta nell'aria di un ambiente condensa e si trasforma in gocce d'acqua; esso viene raggiunto quando l'umidità relativa raggiunge il \SI{100}{\percent}, cioè nell'istante in cui l'aria diventa satura e non riesce più a contenere l'umidità ambientale. Il punto di rugiada si determina tramite l'\textit{approssimazione di Magnus-Tetens}:
\[ T_d = \frac{b \cdot \alpha(T,UR)}{a - \alpha(T,UR)}\]
\[\mbox{con } \alpha(T,UR) = \frac{a \cdot T}{b + T} + \ln(UR) \mbox{, } a = \si{17,27} \mbox{ e } b = \si{237,7}\si{\degreeCelsius}\]
\[T\mbox{ (temperatura misurata): } \SI{0}{\degreeCelsius} < T < \SI{60}{\degreeCelsius}\]
\[UR\mbox{ (umidità relativa): } 0,01 < UR < 1,00\]
La nebbia si forma quando $T = T_d$ e in assenza di vento, mentre la brina quando $T = T_d$ con $T_d < 0$

\paragraph{Modus operandi} La temperatura $T$ e l'umidità relativa $UR$ sono parametri (dalla loro combinazione si ottiene il punto di rugiada $T_d$) che, rispetto alla pressione atmosferica, variano più rapidamente e possono essere impiegati per prevedere la formazione di nebbia o brina in tempi brevi. Pertanto, si ritiene opportuno definire dei modelli stocastici (es. ARMA) per $T$ e $UR$ a scopo predittivo.

\subsection{Vento intenso}
Basandosi sulla \textit{scala Beaufort} (vedi tabella \ref{Beaufort}), un vento può definirsi intenso quando supera i \SI[per-mode=symbol-or-fraction]{50}{\kilo\meter\per\hour}.
\begin{table}[h!]
	\begin{tabular}{|c|l|c|c|}
		\hline
		\textbf{Grado} & \textbf{Descrizione} & \textbf{Velocità} (\si{\knot}s) & \textbf{Velocità} (\si[per-mode=symbol-or-fraction]{\kilo\meter\per\hour})\\ \hline
		0 & Calma & 0 - 1 & 0 - 1 \\ \hline
		1 & Bava di vento & 1 - 3 & 1 - 5 \\ \hline
		2 & Brezza leggera & 4 - 6 & 6 - 11 \\ \hline
		3 & Brezza & 7 - 10 & 12 - 19 \\ \hline
		4 & Brezza vivace & 11 - 16 & 20 - 28 \\ \hline
		5 & Brezza tesa & 17 - 21 & 29 - 38 \\ \hline
		6 & Vento fresco & 22 - 27 & 39 - 49 \\ \hline
		7 & Vento forte & 28 - 33 & 50 - 61 \\ \hline
		8 & Burrasca moderata & 34 - 40 & 62 - 74 \\ \hline
		9 & Burrasca forte & 41 - 47 & 75 - 88 \\ \hline
		10 & Tempesta & 48 - 55 & 89 - 102 \\ \hline
		11 & Fortunale & 56 - 63 & 103 - 117 \\ \hline
		12 & Uragano & $>$ 64 & $>$ 118 \\ \hline
	\end{tabular}
\centering
\caption{scala Beaufort della velocità del vento}
\label{Beaufort}
\end{table}

\paragraph{Modus operandi} Identificazione di un modello stocastico (es. ARMA) per prevedere l'arrivo di un vento con velocità superiore a \SI[per-mode=symbol-or-fraction]{50}{\kilo\meter\per\hour}. 

\subsection{Precipitazioni nevose}
Prevedere se nevicherà o meno risulta essere particolarmente complesso in quanto sono numerosi i fattori fisici e ambientali che devono essere presi in considerazione. Inoltre, sarebbe necessario conoscere lo stato della colonna d'aria sovrastante il suolo, in particolare la temperatura, fino a un'altezza di \SI{800}{\meter}, dati dei quali non si è in possesso. Pertanto, ci si limita a calcolare la \textit{quota neve}. Essa si può determinare a partire dalla \textit{temperatura di bulbo umido} $T_w$, ossia la temperatura più bassa ottenibile dall’evaporazione dell’acqua nell’aria a pressione costante. Essa si calcola così:
\[ T_w = T \cdot (0,25 + 0,006 \cdot UR \cdot \sqrt{\frac{p_{atm}}{1060}})\]
dove \textit{T} è la temperatura, \textit{UR} l'umidità relativa e $p_{atm}$ la pressione atmosferica.
Nota la $T_w$, la quota neve \textit{QN} è data dalla seguente formula:
\[ QN = 10 \cdot \frac{T_w - 1}{0,06}\]
Se, per esempio, ci si trova a \SI{400}{\meter} di quota e si misu, 