\documentclass[
		a4paper,
		cleardoublepage=empty,
		headings=twolinechapter,
		numbers=autoenddot,
]{article}

\usepackage[utf8]{inputenc}
\usepackage[italian]{babel}
\usepackage{import}
\usepackage{todonotes}
\usepackage{color}
\usepackage[normalem]{ulem}


\title{Requisiti}
\author{Lorenzo Leoni, Davide Salvetti, Matteo Verzeroli}
\begin{document}
	\maketitle
	L’obbiettivo è quello di realizzare una applicazione che permetta la gestione delle squadre della Protezione Civile.  
	Gli attori in gioco sono i seguenti: 
	\begin{itemize}
		\item Volontari: coloro che operano durante gli interventi; 
		\item Caposquadra: responsabili di un gruppo di volontari durante gli interventi; 
	\end{itemize}

	I volontari sono organizzati in squadre ed ogni squadra ha un caposquadra che coordina le attività. Ogni squadra si occupa degli interventi all’interno di una specifica area geografica.
	
	Per ogni attore devono essere memorizzate le informazioni anagrafiche di base (nome, cognome, CF…) e il suo stato operativo. Un volontario può dichiararsi operativo quando sta effettivamente partecipando ad un intervento. Il capo squadra ha accesso ad alcune funzionalità particolari, come il tracciamento in tempo reale della posizione degli operatori della sua squadra nel momento in cui sono operativi. 
	
	Gli interventi possono essere programmati (su un calendario) oppure possono essere delle emergenze. Per ogni intervento bisogna memorizzare alcune informazioni di base (tipo di intervento, materiali utilizzati durante l’intervento, partecipanti, ora, luogo...) e poi, al termine, generare un report (in formato PDF) con un riassunto dei dati ed eventuali foto scattate. È possibile, inoltre, visualizzare la lista di interventi.  
	
	Ai volontari verrà richiesto di inserire delle ore di reperibilità. Nel momento in cui viene creato un intervento programmato o un intervento di emergenza, l’applicazione, sulla base delle reperibilità e della difficoltà dell’intervento, crea una squadra formata da un certo numero di volontari e gli invia una notifica. A questo punto i volontari partecipano all’intervento e potranno documentare ciò che è stato fatto, caricare foto, inserire i materiali utilizzati etc. 
	
	L’applicazione fornirà anche la possibilità agli operatori di essere notificati con avvisi riguardanti bollettini meteo, terremoti, allerte della protezione civile nazionali ed eventi particolari provenienti da stazioni installate sul territorio. Le informazioni per generare gli avvisi vengono recuperate tramite alcune API: 
	\begin{itemize}
		\item APRS.FI: sito utilizzato dai radioamatori per la condivisione di dati open. Il sito fornisce alcune API REST da cui è possibile richiedere informazioni su stazioni metereologiche pubbliche;
		\item OpenWeatherMap: il sito permette di scaricare informazioni metereologiche, previsioni e dati sulla qualità dell’aria in tutta Italia tramite delle API REST;
		\item INGV Earthquake event: permette di ricevere informazioni sui terremoti che si sono verificati in una particolare area geografica.
		\item https://www.protezionecivilepop.tk/: web app che fornisce delle API dalla quale è possibile scaricare i bollettini giornalieri della protezione civile (recuperati dalla repository Github della protezione civile) riguardo al rischio idraulico, idrogeologico e temporali;
		\item Recupero di dati da stazioni personalizzate posizionate dalla squadra. La stazione fornirà un API dalla quale poter recuperare dati di interesse (metereologici, vento, livello di torrenti…). 
	\end{itemize}

	Le stazioni personalizzate installate dalla squadra saranno realizzate tramite prototipi Arduino/Rasperry e sensori (reali o simulati) ed esporranno un API REST alla quale richiedere le informazioni mediante un protocollo da noi definito. Ad esempio, si potrà realizzare un sistema di monitoraggio del livello dell’acqua di valli, per essere avvisati in caso di straripamento, oppure dati metereologici (temperatura, umidità, vento, …) per avere informazioni dettagliate su un’area di particolare interesse (per esempio installate in aree protette, risorse naturali). 
	
	Al momento dell’inserimento della squadra verrà richiesto al caposquadra non solo di inserire le informazioni della squadra (partecipanti, area di riferimento, nome, caposquadra etc.) ma anche a quali servizi di notifica si vuole sottoscrivere. Sarà possibile, infatti, richiedere di essere informati sulle condizioni meteo ed altre allerte derivanti da un’analisi dei dati (che sarà effettuata dalla nostra applicazione) di una/più stazione/i accessibili su APRS.FI, specificando il codice della centralina desiderata. Oppure, potrà scegliere di ricevere i bollettini meteo basati sulle previsioni di OpenWeatherMap (che fornisce previsioni più precise ma riguardanti una zona più generica rispetto all’area operativa della squadra). Oppure, in base alla località assegnata alla squadra, ossia il paese di operatività della squadra (specificata in fase di registrazione della squadra), si potrà richiedere di essere avvisati in caso di terremoti in quella zona. Infine, si può richiedere di essere avvisati sui bollettini giornalieri emessi dalla protezione civile nazionale. Indicando poi le API di riferimento delle stazioni personalizzate e la tipologia del dato richiesto (temperatura, livello acqua etc.) sarà possibile essere avvisati su particolari eventi derivanti da un’analisi dei dati ricevuti. 
	
	La nostra applicazione analizzerà i dati metereologici provenienti dalle stazioni metereologiche (su APRS.FI o personali) per generare previsioni a breve termine e avvisare le squadre in caso di particolari fenomeni (vento forte = attenzione alla possibilità di incidenti dovuti alla caduta di piante, temperatura sotto lo zero = possibile ghiaccio, livelli di pioggia elevati = allagamenti, livello acqua fiume elevato = possibilità esondazioni). 
	
	All’interno dell’applicazione verranno anche utilizzate delle mappe (fornite da OpenStreetMap) per visualizzazioni grafiche di posizioni. Inoltre, saranno utilizzati anche delle API (fornite dal sito https://comuni-ita.herokuapp.com/) che permettono di ricevere alcune informazioni (posizione, cap, codice istat etc.) sui comuni italiani.
	
	L’applicazione potrà essere utilizzata su dispositivi mobili o computer, con un’autenticazione dell’utente. 
	
\end{document}