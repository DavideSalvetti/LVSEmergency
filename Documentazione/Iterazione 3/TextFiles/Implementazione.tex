\section{Introduzione}
Nella terza iterazione sono stati implementati i componenti \textit{Data Analyzer} e \textit{Data Collector}, che permettono di raccogliere e analizzare i dati di interesse per la nostra applicazione. Inoltre, sono state implementate le API per la visualizzazione dei dati ambientali raccolti e degli allarmi generati. Infine, è stato implementato il caso d'uso "Visualizzazione posizione real-time" che utilizza delle API di \textit{Open Street Map} per mostrare la posizione degli operatori su una mappa. L'accesso alle API di \textit{Open Street Map} è semplificato dal modulo \textit{Qt Location} fornito da Qt. 

I casi d'uso implementati sono i seguenti:
\begin{itemize}
	\item UC18 - Visualizzazione informazioni zona [astratto];
	\begin{itemize}
		\item UC18.1 - Visualizzazione dati ambientali;
		\item UC18.1 - Visualizzazione allarmi;
	\end{itemize}
	\item UC10 - Visualizzazione posizione real-time
\end{itemize}

\clearpage

\subsection{Data Collector}
\import{./Iterazione 3/TextFiles}{DataCollector.tex}

\clearpage

\import{./Iterazione 3/TextFiles}{EsempioAPIAPRS.tex}

\clearpage

\subsection{Data Analyzer}
\import{./Iterazione 3/TextFiles}{DataAnalyzer.tex}

\clearpage

\subsection{UC18.1 - Visualizzazione dati ambientali}
\textit{Breve descrizione:} il volontario (o caposquadra) accede alla pagina in cui gli vengono mostrati gli ultimi dati disponibili relativi all'area del team di cui fa parte. 
\\
\\
\textit{Attori coinvolti:} volontario, caposquadra, sistema.
\\
\\
\textit{Precondizione:} il volontario (o caposquadra) è loggato nel sistema.
\\
\\
\textit{Postcondizione:} il volontario (o caposquadra) visualizza i dati relativi all'area del team di cui fa parte.
\\
\\
\textit{Procedimento:}
\begin{enumerate}
	\item il volontario (o caposquadra) accede alla pagina "Dati Area".
	\item il sistema mostra gli ultimi dati disponibili relativi a:
	\begin{itemize}
		\item stazioni APRS;
	\end{itemize}
\end{enumerate}

\subsection{UC18.2 - Visualizzazione allarmi}
\textit{Breve descrizione:} il volontario (o caposquadra) accede alla pagina in cui gli vengono mostrati gli allarmi disponibili relativi all'area del team di cui fa parte. 
\\
\\
\textit{Attori coinvolti:} volontario, caposquadra, sistema.
\\
\\
\textit{Precondizione:} il volontario (o caposquadra) è loggato nel sistema.
\\
\\
\textit{Postcondizione:} il volontario (o caposquadra) visualizza gli allarmi relativi all'area del team di cui fa parte.
\\
\\
\textit{Procedimento:}
\begin{enumerate}
	\item il volontario (o caposquadra) accede alla pagina "Dati Area".
	\item il sistema mostra gli allarmi emessi per la specifica area:
	\begin{itemize}
		\item allarme nebbia o brina;
		\item allarme maltempo;
	\end{itemize}
\end{enumerate}

\subsection{UC10 - Visualizzazione posizione real-time}
\textit{Breve descrizione:}il volontario o il caposquadra accedono alla pagina "Mappa" dove possono vedere la posizione real-time dei soli membri operativi della loro squadra su una mappa. 
\\
\\
\textit{Attori coinvolti:} volontario, caposquadra, sistema.
\\
\\
\textit{Precondizione:} il volontario/caposquadra è loggato nel sistema.
\\
\\
\textit{Postcondizione:} il volontario/caposquadra visualizza la posizione dei membri operativi della squadra su una mappa.
\\
\\
\textit{Procedimento:}
\begin{enumerate}
	\item il volontario/caposquadra accede alla pagina "Mappa".
	\item il sistema recupera le informazioni sulla posizione degli utenti operativi della squadra in modo periodico.
	\item il volontario/caposquadra visualizza la posizione dei compagni operativi sulla mappa.
\end{enumerate}